% Options for packages loaded elsewhere
\PassOptionsToPackage{unicode}{hyperref}
\PassOptionsToPackage{hyphens}{url}
\PassOptionsToPackage{dvipsnames,svgnames,x11names}{xcolor}
%
\documentclass[
  letterpaper,
  DIV=11,
  numbers=noendperiod]{scrartcl}

\usepackage{amsmath,amssymb}
\usepackage{iftex}
\ifPDFTeX
  \usepackage[T1]{fontenc}
  \usepackage[utf8]{inputenc}
  \usepackage{textcomp} % provide euro and other symbols
\else % if luatex or xetex
  \usepackage{unicode-math}
  \defaultfontfeatures{Scale=MatchLowercase}
  \defaultfontfeatures[\rmfamily]{Ligatures=TeX,Scale=1}
\fi
\usepackage{lmodern}
\ifPDFTeX\else  
    % xetex/luatex font selection
\fi
% Use upquote if available, for straight quotes in verbatim environments
\IfFileExists{upquote.sty}{\usepackage{upquote}}{}
\IfFileExists{microtype.sty}{% use microtype if available
  \usepackage[]{microtype}
  \UseMicrotypeSet[protrusion]{basicmath} % disable protrusion for tt fonts
}{}
\makeatletter
\@ifundefined{KOMAClassName}{% if non-KOMA class
  \IfFileExists{parskip.sty}{%
    \usepackage{parskip}
  }{% else
    \setlength{\parindent}{0pt}
    \setlength{\parskip}{6pt plus 2pt minus 1pt}}
}{% if KOMA class
  \KOMAoptions{parskip=half}}
\makeatother
\usepackage{xcolor}
\setlength{\emergencystretch}{3em} % prevent overfull lines
\setcounter{secnumdepth}{5}
% Make \paragraph and \subparagraph free-standing
\ifx\paragraph\undefined\else
  \let\oldparagraph\paragraph
  \renewcommand{\paragraph}[1]{\oldparagraph{#1}\mbox{}}
\fi
\ifx\subparagraph\undefined\else
  \let\oldsubparagraph\subparagraph
  \renewcommand{\subparagraph}[1]{\oldsubparagraph{#1}\mbox{}}
\fi

\usepackage{color}
\usepackage{fancyvrb}
\newcommand{\VerbBar}{|}
\newcommand{\VERB}{\Verb[commandchars=\\\{\}]}
\DefineVerbatimEnvironment{Highlighting}{Verbatim}{commandchars=\\\{\}}
% Add ',fontsize=\small' for more characters per line
\usepackage{framed}
\definecolor{shadecolor}{RGB}{241,243,245}
\newenvironment{Shaded}{\begin{snugshade}}{\end{snugshade}}
\newcommand{\AlertTok}[1]{\textcolor[rgb]{0.68,0.00,0.00}{#1}}
\newcommand{\AnnotationTok}[1]{\textcolor[rgb]{0.37,0.37,0.37}{#1}}
\newcommand{\AttributeTok}[1]{\textcolor[rgb]{0.40,0.45,0.13}{#1}}
\newcommand{\BaseNTok}[1]{\textcolor[rgb]{0.68,0.00,0.00}{#1}}
\newcommand{\BuiltInTok}[1]{\textcolor[rgb]{0.00,0.23,0.31}{#1}}
\newcommand{\CharTok}[1]{\textcolor[rgb]{0.13,0.47,0.30}{#1}}
\newcommand{\CommentTok}[1]{\textcolor[rgb]{0.37,0.37,0.37}{#1}}
\newcommand{\CommentVarTok}[1]{\textcolor[rgb]{0.37,0.37,0.37}{\textit{#1}}}
\newcommand{\ConstantTok}[1]{\textcolor[rgb]{0.56,0.35,0.01}{#1}}
\newcommand{\ControlFlowTok}[1]{\textcolor[rgb]{0.00,0.23,0.31}{#1}}
\newcommand{\DataTypeTok}[1]{\textcolor[rgb]{0.68,0.00,0.00}{#1}}
\newcommand{\DecValTok}[1]{\textcolor[rgb]{0.68,0.00,0.00}{#1}}
\newcommand{\DocumentationTok}[1]{\textcolor[rgb]{0.37,0.37,0.37}{\textit{#1}}}
\newcommand{\ErrorTok}[1]{\textcolor[rgb]{0.68,0.00,0.00}{#1}}
\newcommand{\ExtensionTok}[1]{\textcolor[rgb]{0.00,0.23,0.31}{#1}}
\newcommand{\FloatTok}[1]{\textcolor[rgb]{0.68,0.00,0.00}{#1}}
\newcommand{\FunctionTok}[1]{\textcolor[rgb]{0.28,0.35,0.67}{#1}}
\newcommand{\ImportTok}[1]{\textcolor[rgb]{0.00,0.46,0.62}{#1}}
\newcommand{\InformationTok}[1]{\textcolor[rgb]{0.37,0.37,0.37}{#1}}
\newcommand{\KeywordTok}[1]{\textcolor[rgb]{0.00,0.23,0.31}{#1}}
\newcommand{\NormalTok}[1]{\textcolor[rgb]{0.00,0.23,0.31}{#1}}
\newcommand{\OperatorTok}[1]{\textcolor[rgb]{0.37,0.37,0.37}{#1}}
\newcommand{\OtherTok}[1]{\textcolor[rgb]{0.00,0.23,0.31}{#1}}
\newcommand{\PreprocessorTok}[1]{\textcolor[rgb]{0.68,0.00,0.00}{#1}}
\newcommand{\RegionMarkerTok}[1]{\textcolor[rgb]{0.00,0.23,0.31}{#1}}
\newcommand{\SpecialCharTok}[1]{\textcolor[rgb]{0.37,0.37,0.37}{#1}}
\newcommand{\SpecialStringTok}[1]{\textcolor[rgb]{0.13,0.47,0.30}{#1}}
\newcommand{\StringTok}[1]{\textcolor[rgb]{0.13,0.47,0.30}{#1}}
\newcommand{\VariableTok}[1]{\textcolor[rgb]{0.07,0.07,0.07}{#1}}
\newcommand{\VerbatimStringTok}[1]{\textcolor[rgb]{0.13,0.47,0.30}{#1}}
\newcommand{\WarningTok}[1]{\textcolor[rgb]{0.37,0.37,0.37}{\textit{#1}}}

\providecommand{\tightlist}{%
  \setlength{\itemsep}{0pt}\setlength{\parskip}{0pt}}\usepackage{longtable,booktabs,array}
\usepackage{calc} % for calculating minipage widths
% Correct order of tables after \paragraph or \subparagraph
\usepackage{etoolbox}
\makeatletter
\patchcmd\longtable{\par}{\if@noskipsec\mbox{}\fi\par}{}{}
\makeatother
% Allow footnotes in longtable head/foot
\IfFileExists{footnotehyper.sty}{\usepackage{footnotehyper}}{\usepackage{footnote}}
\makesavenoteenv{longtable}
\usepackage{graphicx}
\makeatletter
\def\maxwidth{\ifdim\Gin@nat@width>\linewidth\linewidth\else\Gin@nat@width\fi}
\def\maxheight{\ifdim\Gin@nat@height>\textheight\textheight\else\Gin@nat@height\fi}
\makeatother
% Scale images if necessary, so that they will not overflow the page
% margins by default, and it is still possible to overwrite the defaults
% using explicit options in \includegraphics[width, height, ...]{}
\setkeys{Gin}{width=\maxwidth,height=\maxheight,keepaspectratio}
% Set default figure placement to htbp
\makeatletter
\def\fps@figure{htbp}
\makeatother

\KOMAoption{captions}{tableheading}
\makeatletter
\@ifpackageloaded{caption}{}{\usepackage{caption}}
\AtBeginDocument{%
\ifdefined\contentsname
  \renewcommand*\contentsname{Table of contents}
\else
  \newcommand\contentsname{Table of contents}
\fi
\ifdefined\listfigurename
  \renewcommand*\listfigurename{List of Figures}
\else
  \newcommand\listfigurename{List of Figures}
\fi
\ifdefined\listtablename
  \renewcommand*\listtablename{List of Tables}
\else
  \newcommand\listtablename{List of Tables}
\fi
\ifdefined\figurename
  \renewcommand*\figurename{Figure}
\else
  \newcommand\figurename{Figure}
\fi
\ifdefined\tablename
  \renewcommand*\tablename{Table}
\else
  \newcommand\tablename{Table}
\fi
}
\@ifpackageloaded{float}{}{\usepackage{float}}
\floatstyle{ruled}
\@ifundefined{c@chapter}{\newfloat{codelisting}{h}{lop}}{\newfloat{codelisting}{h}{lop}[chapter]}
\floatname{codelisting}{Listing}
\newcommand*\listoflistings{\listof{codelisting}{List of Listings}}
\makeatother
\makeatletter
\makeatother
\makeatletter
\@ifpackageloaded{caption}{}{\usepackage{caption}}
\@ifpackageloaded{subcaption}{}{\usepackage{subcaption}}
\makeatother
\ifLuaTeX
  \usepackage{selnolig}  % disable illegal ligatures
\fi
\usepackage{bookmark}

\IfFileExists{xurl.sty}{\usepackage{xurl}}{} % add URL line breaks if available
\urlstyle{same} % disable monospaced font for URLs
\hypersetup{
  pdftitle={Running A Single Prevalence Analysis},
  colorlinks=true,
  linkcolor={blue},
  filecolor={Maroon},
  citecolor={Blue},
  urlcolor={Blue},
  pdfcreator={LaTeX via pandoc}}

\title{Running A Single Prevalence Analysis}
\author{}
\date{}

\begin{document}
\maketitle

\renewcommand*\contentsname{Table of contents}
{
\hypersetup{linkcolor=}
\setcounter{tocdepth}{3}
\tableofcontents
}
\subsection{Overview}\label{overview}

This tutorial walks you through running a \texttt{CohortPrevalence} for
a single analysis with the simplest yearly prevalence settings.

\subsection{Connection Details}\label{connection-details}

We use the \texttt{ClinicalCharacteristics} package to specify execution
settings for the analysis with information on our databases and schemas
of interest, and connect to the databases using
\texttt{DatabaseConnector}.

\begin{Shaded}
\begin{Highlighting}[]
\NormalTok{configBlock }\OtherTok{\textless{}{-}} \StringTok{"[Add config block]"}
\NormalTok{connectionDetails }\OtherTok{\textless{}{-}}\NormalTok{ DatabaseConnector}\SpecialCharTok{::}\FunctionTok{createConnectionDetails}\NormalTok{(}
  \AttributeTok{dbms =}\NormalTok{ config}\SpecialCharTok{::}\FunctionTok{get}\NormalTok{(}\StringTok{"dbms"}\NormalTok{, }\AttributeTok{config =}\NormalTok{ configBlock),}
  \AttributeTok{user =}\NormalTok{ config}\SpecialCharTok{::}\FunctionTok{get}\NormalTok{(}\StringTok{"user"}\NormalTok{, }\AttributeTok{config =}\NormalTok{ configBlock),}
  \AttributeTok{password =}\NormalTok{ config}\SpecialCharTok{::}\FunctionTok{get}\NormalTok{(}\StringTok{"password"}\NormalTok{, }\AttributeTok{config =}\NormalTok{ configBlock),}
  \AttributeTok{connectionString =}\NormalTok{ config}\SpecialCharTok{::}\FunctionTok{get}\NormalTok{(}\StringTok{"connectionString"}\NormalTok{, }\AttributeTok{config =}\NormalTok{ configBlock)}
\NormalTok{)}

\NormalTok{executionSettings }\OtherTok{\textless{}{-}}\NormalTok{ ClinicalCharacteristics}\SpecialCharTok{::}\FunctionTok{createExecutionSettings}\NormalTok{(}
  \AttributeTok{connectionDetails =}\NormalTok{ connectionDetails,}
  \AttributeTok{cdmDatabaseSchema =}\NormalTok{ cdmDatabaseSchema, }\CommentTok{\# schema containing patient data}
  \AttributeTok{workDatabaseSchema =}\NormalTok{ workDatabaseSchema, }\CommentTok{\# schema to write to}
  \AttributeTok{tempEmulationSchema =}\NormalTok{ tempEmulationSchema, }\CommentTok{\# schema to write temporary tables to}
  \AttributeTok{cohortTable =}\NormalTok{ cohortTable, }\CommentTok{\# table on the workDatabaseSchema containing cohort data}
  \AttributeTok{cdmSourceName =}\NormalTok{ cdmSourceName }\CommentTok{\# human{-}readable database source name}
\NormalTok{)}

\NormalTok{connection }\OtherTok{\textless{}{-}}\NormalTok{ DatabaseConnector}\SpecialCharTok{::}\FunctionTok{connect}\NormalTok{(connectionDetails)}
\end{Highlighting}
\end{Shaded}

\subsection{Defining cohorts}\label{defining-cohorts}

In this test case, we use \texttt{CapR} to generate a cohort of
hypertension patients in our OMOP database.

\begin{Shaded}
\begin{Highlighting}[]
\FunctionTok{library}\NormalTok{(Capr)}

\NormalTok{hypertensiveDisorder }\OtherTok{\textless{}{-}} \FunctionTok{cs}\NormalTok{(}
  \FunctionTok{descendants}\NormalTok{(}\DecValTok{316866}\NormalTok{),}
  \AttributeTok{name =} \StringTok{"Hypertensive disorder"}
\NormalTok{)}

\NormalTok{hypertensiveDisorder }\OtherTok{\textless{}{-}} \FunctionTok{getConceptSetDetails}\NormalTok{(}
  \AttributeTok{x =}\NormalTok{ hypertensiveDisorder, }
  \AttributeTok{con =}\NormalTok{ connection, }
  \AttributeTok{vocabularyDatabaseSchema =}\NormalTok{ executionSettings}\SpecialCharTok{$}\NormalTok{cdmDatabaseSchema)}

\NormalTok{cohort }\OtherTok{\textless{}{-}} \FunctionTok{cohort}\NormalTok{(}
  \AttributeTok{entry =} \FunctionTok{entry}\NormalTok{(}
    \FunctionTok{conditionOccurrence}\NormalTok{(}\AttributeTok{conceptSet =}\NormalTok{ hypertensiveDisorder),}
    \AttributeTok{primaryCriteriaLimit =} \StringTok{"All"}
\NormalTok{  ),}
  \AttributeTok{exit =} \FunctionTok{exit}\NormalTok{(}
    \AttributeTok{endStrategy =} \FunctionTok{observationExit}\NormalTok{()}
\NormalTok{  )}
\NormalTok{)}

\NormalTok{json }\OtherTok{\textless{}{-}} \FunctionTok{as.json}\NormalTok{(cohort)}
\NormalTok{sql }\OtherTok{\textless{}{-}}\NormalTok{ CirceR}\SpecialCharTok{::}\FunctionTok{buildCohortQuery}\NormalTok{(}
  \AttributeTok{expression =}\NormalTok{ CirceR}\SpecialCharTok{::}\FunctionTok{cohortExpressionFromJson}\NormalTok{(json),}
  \AttributeTok{options =}\NormalTok{ CirceR}\SpecialCharTok{::}\FunctionTok{createGenerateOptions}\NormalTok{(}\AttributeTok{generateStats =} \ConstantTok{FALSE}\NormalTok{)}
\NormalTok{)}

\NormalTok{cohortDefinitionSet }\OtherTok{\textless{}{-}} \FunctionTok{data.frame}\NormalTok{(}
  \AttributeTok{cohortId =} \DecValTok{316866}\NormalTok{,}
  \AttributeTok{cohortName =} \StringTok{"hypertension"}\NormalTok{,}
  \AttributeTok{json =}\NormalTok{ json,}
  \AttributeTok{sql =}\NormalTok{ sql}
\NormalTok{)}

\NormalTok{cohortTableNames }\OtherTok{\textless{}{-}}\NormalTok{ CohortGenerator}\SpecialCharTok{::}\FunctionTok{getCohortTableNames}\NormalTok{(}\AttributeTok{cohortTable =}\NormalTok{ executionSettings}\SpecialCharTok{$}\NormalTok{cohortTable)}

\NormalTok{CohortGenerator}\SpecialCharTok{::}\FunctionTok{createCohortTables}\NormalTok{(}
  \AttributeTok{connectionDetails =}\NormalTok{ connectionDetails,}
  \AttributeTok{cohortDatabaseSchema =}\NormalTok{ executionSettings}\SpecialCharTok{$}\NormalTok{workDatabaseSchema,}
  \AttributeTok{cohortTableNames =}\NormalTok{ cohortTableNames}
\NormalTok{)}
\NormalTok{CohortGenerator}\SpecialCharTok{::}\FunctionTok{generateCohortSet}\NormalTok{(}
  \AttributeTok{connectionDetails =}\NormalTok{ connectionDetails,}
  \AttributeTok{cohortDatabaseSchema =}\NormalTok{ executionSettings}\SpecialCharTok{$}\NormalTok{workDatabaseSchema,}
  \AttributeTok{tempEmulationSchema =}\NormalTok{ executionSettings}\SpecialCharTok{$}\NormalTok{tempEmulationSchema,}
  \AttributeTok{cohortTableNames =}\NormalTok{ cohortTableNames,}
  \AttributeTok{cdmDatabaseSchema =}\NormalTok{ executionSettings}\SpecialCharTok{$}\NormalTok{cdmDatabaseSchema,}
  \AttributeTok{cohortDefinitionSet =}\NormalTok{ cohortDefinitionSet}
\NormalTok{)}

\CommentTok{\# Check counts {-}{-}{-}{-}{-}{-}{-}{-}{-}{-}{-}{-}{-}{-}{-}{-}{-}{-}{-}{-}{-}{-}{-}{-}{-}{-}{-}{-}{-}{-}{-}{-}{-}{-}}
\NormalTok{cohortCounts }\OtherTok{\textless{}{-}}\NormalTok{ CohortGenerator}\SpecialCharTok{::}\FunctionTok{getCohortCounts}\NormalTok{(}
  \AttributeTok{connectionDetails =}\NormalTok{ connectionDetails,}
  \AttributeTok{cohortDatabaseSchema =}\NormalTok{ executionSettings}\SpecialCharTok{$}\NormalTok{workDatabaseSchema,}
  \AttributeTok{cohortTable =}\NormalTok{ cohortTableNames}\SpecialCharTok{$}\NormalTok{cohortTable}
\NormalTok{)}
\NormalTok{cohortCounts}
\end{Highlighting}
\end{Shaded}

\subsection{Step 1: Cohorts and periods of
interest}\label{step-1-cohorts-and-periods-of-interest}

The first step of the analysis is to specify the cohort of prevalent
interest and the yearly periods of interest. In order to do so, we
create the \texttt{R6} classes that define these analyses settings.
Sometimes, we may want to do an analysis on a subpopulation of the
overall database (i.e.~prevalence of hypertension in sitagliptin users).
In this analysis, we care about the entire population, and so we leave
the \texttt{populationCohort} option \texttt{NULL}.

\begin{Shaded}
\begin{Highlighting}[]
\NormalTok{prevalentCohort }\OtherTok{\textless{}{-}} \FunctionTok{createPrevalenceCohort}\NormalTok{(}\AttributeTok{cohortId =} \DecValTok{316866}\NormalTok{,}
                                          \AttributeTok{cohortName =} \StringTok{"Hypertension"}\NormalTok{)}
\NormalTok{periodOfInterest }\OtherTok{\textless{}{-}} \FunctionTok{createYearlyPrevalence}\NormalTok{(}\AttributeTok{range =} \FunctionTok{c}\NormalTok{(}\DecValTok{2016}\SpecialCharTok{:}\DecValTok{2020}\NormalTok{))}
\NormalTok{populationCohort }\OtherTok{\textless{}{-}} \ConstantTok{NULL}
\end{Highlighting}
\end{Shaded}

\subsection{Step 2: Prevalence analysis
options}\label{step-2-prevalence-analysis-options}

Next, we define the options specific to the prevalence analysis. This
includes a choice of numerator and denominator computational technique.
\texttt{CohortPrevalence} uses operational definitions of prevalence
from \href{https://pubmed.ncbi.nlm.nih.gov/30588119/}{Rassen et al}.
Please see the vignette \texttt{prevalence} for definitions of the
numerator and denominator choices.

Beyond the numerator and denominator, we also create options for the
length of lookback (and whether we only want to use observed time in the
lookback), rate multiplier, minimum observation length, strata variables
(default age and gender), and whether or not we want to use only the
first observation period,

\begin{Shaded}
\begin{Highlighting}[]
\NormalTok{analysisId }\OtherTok{\textless{}{-}} \DecValTok{123} \CommentTok{\# Any unique integer ID to define this analysis}

\CommentTok{\# Select numerator and denominator options}
\NormalTok{numeratorType }\OtherTok{\textless{}{-}} \StringTok{"pn1"} \CommentTok{\#pn1 or pn2}
\NormalTok{denominatorType }\OtherTok{\textless{}{-}} \FunctionTok{createDenominatorType}\NormalTok{(}\AttributeTok{denomType =} \StringTok{"pd3"}\NormalTok{) }\CommentTok{\#pd1, pd2, pd3, or pd4}

\CommentTok{\# Set lookback period options}
\NormalTok{lookBackOptions }\OtherTok{\textless{}{-}} \FunctionTok{createLookBackOptions}\NormalTok{(}\AttributeTok{lookBackDays =} \DecValTok{99999}\NormalTok{L,}
                                         \AttributeTok{useObservedTimeOnly =} \ConstantTok{FALSE}\NormalTok{)}

\CommentTok{\# Set strata options {-} NULL by default uses age and gender}
\NormalTok{strata }\OtherTok{=} \ConstantTok{NULL}

\CommentTok{\# Set other specifications}
\NormalTok{minimumObservationLength }\OtherTok{=} \DecValTok{0}\NormalTok{L}
\NormalTok{useOnlyFirstObservationPeriod }\OtherTok{=} \ConstantTok{FALSE}
\NormalTok{multiplier }\OtherTok{=} \DecValTok{100000}

\NormalTok{prevalenceAnalysisClass }\OtherTok{\textless{}{-}} \FunctionTok{createCohortPrevalenceAnalysis}\NormalTok{(}
  \AttributeTok{analysisId =}\NormalTok{ analysisId,}
  \AttributeTok{prevalentCohort =}\NormalTok{ prevalentCohort,}
  \AttributeTok{periodOfInterest =}\NormalTok{ periodOfInterest,}
  \AttributeTok{lookBackOptions =}\NormalTok{ lookBackOptions,}
  \AttributeTok{numeratorType =}\NormalTok{ numeratorType,}
  \AttributeTok{denominatorType =}\NormalTok{ denominatorType,}
  \AttributeTok{minimumObservationLength =}\NormalTok{  minimumObservationLength,}
  \AttributeTok{useOnlyFirstObservationPeriod =}\NormalTok{  useOnlyFirstObservationPeriod,}
  \AttributeTok{multiplier =}\NormalTok{ multiplier,}
  \AttributeTok{strata =}\NormalTok{ strata,}
  \AttributeTok{populationCohort =}\NormalTok{ populationCohort)}
\end{Highlighting}
\end{Shaded}

\subsection{Step 3: Run analyses}\label{step-3-run-analyses}

The \texttt{prevalenceAnalysisClass} object is now a wrapped-up package
of all our analysis specifications. Using this object, we can now run
the analysis with \texttt{generateSinglePrevalence} and our previously
defined database connection settings.

\begin{Shaded}
\begin{Highlighting}[]
\CommentTok{\# Results}
\NormalTok{results }\OtherTok{\textless{}{-}} \FunctionTok{generateSinglePrevalence}\NormalTok{(}\AttributeTok{prevalenceAnalysisClass =}\NormalTok{ prevalenceAnalysisClass,}
                                    \AttributeTok{executionSettings =}\NormalTok{ executionSettings,}
                                    \AttributeTok{connection =}\NormalTok{ connection)}
\end{Highlighting}
\end{Shaded}

\subsection{Step 4: Export}\label{step-4-export}

Now, we can write the results dataframe into \texttt{.csv} format for
sharing results with \texttt{exportPrevalenceResults} Sometimes, we want
to externally review the Sql queries used to generate the prevalence
objects. We can do this with \texttt{exportPrevalenceQuery}.

\begin{Shaded}
\begin{Highlighting}[]
\NormalTok{outputFolder }\OtherTok{\textless{}{-}}\NormalTok{ here}\SpecialCharTok{::}\FunctionTok{here}\NormalTok{(}\StringTok{"results"}\NormalTok{)}

\CommentTok{\# Export results to CSV format}
\FunctionTok{exportPrevalenceResults}\NormalTok{(results,}
                        \AttributeTok{outputFolder =}\NormalTok{ outputFolder)}

\CommentTok{\# Save SQL query}
\FunctionTok{exportPrevalenceQuery}\NormalTok{(prevalenceAnalysisClass,}
                      \AttributeTok{outputFolder =}\NormalTok{ outputFolder)}
\end{Highlighting}
\end{Shaded}




\end{document}
